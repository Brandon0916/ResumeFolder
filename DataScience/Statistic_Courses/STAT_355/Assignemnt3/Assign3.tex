\documentclass[11pt,]{article}
\usepackage{lmodern}
\usepackage{amssymb,amsmath}
\usepackage{ifxetex,ifluatex}
\usepackage{fixltx2e} % provides \textsubscript
\ifnum 0\ifxetex 1\fi\ifluatex 1\fi=0 % if pdftex
  \usepackage[T1]{fontenc}
  \usepackage[utf8]{inputenc}
\else % if luatex or xelatex
  \ifxetex
    \usepackage{mathspec}
  \else
    \usepackage{fontspec}
  \fi
  \defaultfontfeatures{Ligatures=TeX,Scale=MatchLowercase}
\fi
% use upquote if available, for straight quotes in verbatim environments
\IfFileExists{upquote.sty}{\usepackage{upquote}}{}
% use microtype if available
\IfFileExists{microtype.sty}{%
\usepackage{microtype}
\UseMicrotypeSet[protrusion]{basicmath} % disable protrusion for tt fonts
}{}
\usepackage[left=1.5cm,right=1.5cm,top=2cm,bottom=3cm]{geometry}
\usepackage{hyperref}
\hypersetup{unicode=true,
            pdftitle={Stat355, Assignment \#3, due: Friday, October 18 in class},
            pdfauthor={Put your name},
            pdfborder={0 0 0},
            breaklinks=true}
\urlstyle{same}  % don't use monospace font for urls
\usepackage{color}
\usepackage{fancyvrb}
\newcommand{\VerbBar}{|}
\newcommand{\VERB}{\Verb[commandchars=\\\{\}]}
\DefineVerbatimEnvironment{Highlighting}{Verbatim}{commandchars=\\\{\}}
% Add ',fontsize=\small' for more characters per line
\usepackage{framed}
\definecolor{shadecolor}{RGB}{248,248,248}
\newenvironment{Shaded}{\begin{snugshade}}{\end{snugshade}}
\newcommand{\KeywordTok}[1]{\textcolor[rgb]{0.13,0.29,0.53}{\textbf{#1}}}
\newcommand{\DataTypeTok}[1]{\textcolor[rgb]{0.13,0.29,0.53}{#1}}
\newcommand{\DecValTok}[1]{\textcolor[rgb]{0.00,0.00,0.81}{#1}}
\newcommand{\BaseNTok}[1]{\textcolor[rgb]{0.00,0.00,0.81}{#1}}
\newcommand{\FloatTok}[1]{\textcolor[rgb]{0.00,0.00,0.81}{#1}}
\newcommand{\ConstantTok}[1]{\textcolor[rgb]{0.00,0.00,0.00}{#1}}
\newcommand{\CharTok}[1]{\textcolor[rgb]{0.31,0.60,0.02}{#1}}
\newcommand{\SpecialCharTok}[1]{\textcolor[rgb]{0.00,0.00,0.00}{#1}}
\newcommand{\StringTok}[1]{\textcolor[rgb]{0.31,0.60,0.02}{#1}}
\newcommand{\VerbatimStringTok}[1]{\textcolor[rgb]{0.31,0.60,0.02}{#1}}
\newcommand{\SpecialStringTok}[1]{\textcolor[rgb]{0.31,0.60,0.02}{#1}}
\newcommand{\ImportTok}[1]{#1}
\newcommand{\CommentTok}[1]{\textcolor[rgb]{0.56,0.35,0.01}{\textit{#1}}}
\newcommand{\DocumentationTok}[1]{\textcolor[rgb]{0.56,0.35,0.01}{\textbf{\textit{#1}}}}
\newcommand{\AnnotationTok}[1]{\textcolor[rgb]{0.56,0.35,0.01}{\textbf{\textit{#1}}}}
\newcommand{\CommentVarTok}[1]{\textcolor[rgb]{0.56,0.35,0.01}{\textbf{\textit{#1}}}}
\newcommand{\OtherTok}[1]{\textcolor[rgb]{0.56,0.35,0.01}{#1}}
\newcommand{\FunctionTok}[1]{\textcolor[rgb]{0.00,0.00,0.00}{#1}}
\newcommand{\VariableTok}[1]{\textcolor[rgb]{0.00,0.00,0.00}{#1}}
\newcommand{\ControlFlowTok}[1]{\textcolor[rgb]{0.13,0.29,0.53}{\textbf{#1}}}
\newcommand{\OperatorTok}[1]{\textcolor[rgb]{0.81,0.36,0.00}{\textbf{#1}}}
\newcommand{\BuiltInTok}[1]{#1}
\newcommand{\ExtensionTok}[1]{#1}
\newcommand{\PreprocessorTok}[1]{\textcolor[rgb]{0.56,0.35,0.01}{\textit{#1}}}
\newcommand{\AttributeTok}[1]{\textcolor[rgb]{0.77,0.63,0.00}{#1}}
\newcommand{\RegionMarkerTok}[1]{#1}
\newcommand{\InformationTok}[1]{\textcolor[rgb]{0.56,0.35,0.01}{\textbf{\textit{#1}}}}
\newcommand{\WarningTok}[1]{\textcolor[rgb]{0.56,0.35,0.01}{\textbf{\textit{#1}}}}
\newcommand{\AlertTok}[1]{\textcolor[rgb]{0.94,0.16,0.16}{#1}}
\newcommand{\ErrorTok}[1]{\textcolor[rgb]{0.64,0.00,0.00}{\textbf{#1}}}
\newcommand{\NormalTok}[1]{#1}
\usepackage{graphicx,grffile}
\makeatletter
\def\maxwidth{\ifdim\Gin@nat@width>\linewidth\linewidth\else\Gin@nat@width\fi}
\def\maxheight{\ifdim\Gin@nat@height>\textheight\textheight\else\Gin@nat@height\fi}
\makeatother
% Scale images if necessary, so that they will not overflow the page
% margins by default, and it is still possible to overwrite the defaults
% using explicit options in \includegraphics[width, height, ...]{}
\setkeys{Gin}{width=\maxwidth,height=\maxheight,keepaspectratio}
\setlength{\emergencystretch}{3em}  % prevent overfull lines
\providecommand{\tightlist}{%
  \setlength{\itemsep}{0pt}\setlength{\parskip}{0pt}}
\setcounter{secnumdepth}{0}
% Redefines (sub)paragraphs to behave more like sections
\ifx\paragraph\undefined\else
\let\oldparagraph\paragraph
\renewcommand{\paragraph}[1]{\oldparagraph{#1}\mbox{}}
\fi
\ifx\subparagraph\undefined\else
\let\oldsubparagraph\subparagraph
\renewcommand{\subparagraph}[1]{\oldsubparagraph{#1}\mbox{}}
\fi

%%% Use protect on footnotes to avoid problems with footnotes in titles
\let\rmarkdownfootnote\footnote%
\def\footnote{\protect\rmarkdownfootnote}

%%% Change title format to be more compact
\usepackage{titling}

% Create subtitle command for use in maketitle
\newcommand{\subtitle}[1]{
  \posttitle{
    \begin{center}\large#1\end{center}
    }
}

\setlength{\droptitle}{-2em}
  \title{Stat355, Assignment \#3, due: Friday, October 18 in class}
  \pretitle{\vspace{\droptitle}\centering\huge}
  \posttitle{\par}
  \author{Put your name}
  \preauthor{\centering\large\emph}
  \postauthor{\par}
  \predate{\centering\large\emph}
  \postdate{\par}
  \date{2019-10-16}


\begin{document}
\maketitle

\subsection{Multiple regression and Dummy Variables, Section
11.2}\label{multiple-regression-and-dummy-variables-section-11.2}

\subsubsection{Instructions:}\label{instructions}

\begin{enumerate}
\def\labelenumi{\arabic{enumi}.}
\tightlist
\item
  Complete your assignment in R Markdown using this file as a template.
  Insert R code in the R chunks, and type in your response after the
  corresponding R chunk leaving one blank line between the R chunk and
  your comments.
\item
  Execute each line of code separately to ensure that it works properly.
\item
  Either {[}knit the entire document to pdf{]} or {[}knit to HTML or
  Word and print to pdf{]}.
\item
  Submit the pdf file to CourseSpaces in the Assignment 3 activity.
\end{enumerate}

\subsubsection{Data Description:}\label{data-description}

Question 11.Review.15(new book) 11.Review.17(old book)

In a study, subjects were preterm infant with low birth weights born in
three different hospitals. The variables are:

\begin{verbatim}
WEIGHT:  weight in kg  (Y variable)  
WEEKS:  gestation age in weeks  
HOSP:  Hospital of birth, A, B or C  
\end{verbatim}

\begin{enumerate}
\def\labelenumi{\arabic{enumi}.}
\setcounter{enumi}{-1}
\tightlist
\item
  Read the data into R using the read.csv function.
\end{enumerate}

\begin{Shaded}
\begin{Highlighting}[]
\NormalTok{knitr}\OperatorTok{::}\NormalTok{opts_chunk}\OperatorTok{$}\KeywordTok{set}\NormalTok{(}\DataTypeTok{fig.width=}\DecValTok{8}\NormalTok{, }\DataTypeTok{fig.height=}\DecValTok{6}\NormalTok{) }\CommentTok{#set size of graphs}
\NormalTok{HOSPB<-}\KeywordTok{read.csv}\NormalTok{(}\StringTok{'REV_C11_17.csv'}\NormalTok{)}
\KeywordTok{dim}\NormalTok{(HOSPB)}
\end{Highlighting}
\end{Shaded}

\begin{verbatim}
## [1] 40  3
\end{verbatim}

\begin{Shaded}
\begin{Highlighting}[]
\NormalTok{HOSPB}\OperatorTok{$}\NormalTok{HOSPn <-}\StringTok{ }\KeywordTok{as.numeric}\NormalTok{(HOSPB}\OperatorTok{$}\NormalTok{HOSP)  }\CommentTok{#create a numeric HOSP for plotting}
\end{Highlighting}
\end{Shaded}

Questions:

\begin{enumerate}
\def\labelenumi{\arabic{enumi}.}
\item
  Provide descriptive statistics and a pairs plot for the data.\\
  Comment on your results and especially on any unusual features in the
  data. (3 marks)
\item
  Provide summary statistics of the dataset by HOSP and comment. (3
  marks)
\item
  Provide a scatterplot of WEIGHT versus WEEKS, using a different
  plotting character and colour for each of the three hospitals. Comment
  on the graph. (2 marks)
\item
  Fit parallel lines model for WEIGHT versus WEEKS, where the intercepts
  may differ by HOSP but the slopes are the same. Provide a summary of
  the model and comment. (4 marks)
\item
  Fit a model for WEIGHT versus WEEKS, where the intercepts and slopes
  may differ by HOSP. Provide a summary of the model and comment. (4
  marks)
\item
  Provide a scatterplot of WEIGHT versus WEEKS, using a different
  plotting character and colour for each of the three hospitals. Overlay
  the fitted lines from the model in Question 5. (2 marks)
\item
  Compare the models in Questions 4 and 5. Which one model would you
  present to your boss and why? (3 marks)
\end{enumerate}


\end{document}
